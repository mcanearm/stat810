\documentclass[12pt]{article}
\usepackage{fullpage,hyperref}\setlength{\parskip}{3mm}\setlength{\parindent}{0mm}
\begin{document}

\begin{center}\bf
Homework 6. Due by 11:59pm on Sunday 10/19.

Collaborative research \& Human participants and animal subjects

\end{center}
Most statisticians do not have to worry directly about running large scientific research groups and dealing with the bureaucracy of ethical data collection. Yet, most statisticians collaborate sometimes with scientists who do. Read pages 24--28, 39--42 and 48--49 of {\em On Being a Scientist}.  Write brief answers to the following questions, by editing the tex file available at \url{https://github.com/ionides/810f25}, and submit the resulting pdf file via Canvas.

\begin{enumerate}

\item What is an IRB? Does a project studying aggregated observational data on human subjects (say, the total number of road accident injuries per state per year) need IRB approval to receive federal funding?

IRB is definitely not required for aggregated data, especially data which cannot be tied back 
to individuals. For individual subjects it is required in research contexts for publication.

\item Suggest some ingredients which could lead to successful collaboration between two statisticians and/or between a statistician and a scientist.

This depends a lot on the research topic in questions. For two statisticians, I think close overlap
is very important for collaboration, as you can both contribute equally. For more applied work across
disciplines, having a domain expert alongside the statistical expertise is beneficial for ensuring
that the "right" problems are getting solved.

\item Collaborative group sizes can be small or large. Identify some strengths and weaknesses of larger collaborative groups relative to smaller collaborative groups.

Larger groups offer broader perspectives, but too many collaborators can make it difficult to contribute equally and dilute contributions from members. 

With too few collaborators, there might be a large amount of labor that, even when split equally, is
too much for a single person to accomplish in a reasonable amount of time. Smaller groups are also 
somewhat limited in scope to smaller projects.

 
\item \label{p1} What are the advantages and disadvantages of being a conscientious collaborator who makes careful, thoughtful but timely contributions to the project, reads widely and takes the time to understand as much of the project as possible.

The obvious advantage is that you are a better teammate and your contributions are higher quality. It 
has a surprising number of disadvantages, however - first off, it slows down collaboration. Second, 
there could be a sort of "minimum level of quality" contribution. Perhaps that level of collaboration
needed does not require your perception of required contribution.

Still, it'd be better to err on the side of high quality contributions, as the only thing slower
than slow, but high quality contributions, is rapid (but wrong) contributions that you have to correct later.

\item Would you expect a PhD thesis adviser to act like the conscientious collaborator of question~\ref{p1} on your own thesis research? 

Honestly, not really - I expect that my PhD thesis advisor is busy with their own research and with other students, so it's on me to maintain and be responsible for the quality of my thesis with some secondary review and guidance from my advisor. You rely on your advisor for advice, not for labor.

\item What are some advantages and disadvantages of joining a project and then making a minimal contribution? Can this be responsible behavior? Consider the following example: you help a scientist carry out a statistical procedure and you help write up the paragraph describing it; you accept coauthorship on the resulting paper, without spending time on all other aspects of the paper.

I think for statistics this is actually a pretty constructive and expected level of contribution. Statistics is a support field, so supporting other scientists in this way is a beneficial and direct
way to contribute to scientific endeavors. And you get an authorship out of it! 

That being said, as a coauthor, you're still somewhat responsible for the remaining contents of the paper, so you need to be familiar with the work enough that your own reputation isn't affected
by poor results.

\item How can one maintain a reasonable level of agreement within a collaboration on the expected involvement of each collaborator?

Open communication is the key - at the start, delineating who's doing what, and as the project moves on, continuing to check in and adjusting priorities and tasks. But ultimately, you can 
accomplish anything if you're willing to do 100\% of the work and give away 100\% of the credit.

\end{enumerate}
\end{document}
