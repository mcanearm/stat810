\documentclass[12pt]{article}
\usepackage{fullpage,hyperref}\setlength{\parskip}{3mm}\setlength{\parindent}{0mm}
\begin{document}

\begin{center}\bf
Homework 5. Due by 11:59pm on Sunday 10/5.

Conflicts of interest and conflicts of commitment

\end{center}
Conflicts of interest (COI) and conflicts of commitment (COC) are a central topic for maintaining responsible scientific conduct since failure to manage COI/COC situations can lead to irresponsible conduct. Read pages 43--47 of {\em On Being a Scientist}. Write brief answers to the following questions, by editing the tex file available at \url{https://github.com/ionides/810f25}, and submit the resulting pdf file via Canvas.

\begin{enumerate}

\item What is the difference between a conflict of interest and a conflict of commitments? 

Conflicts of interest are when you have competing interests over a particular project or issue, or when you have a vested interest in an outcome for which you should be impartial. Conflicts of commitment, by contrast, are more along the lines of time management and deciding who/what gets attention. It is not the same in that it is not an ethical dilemma. Effectively, the difference is being compromised versus being busy.

\item Is there a clear delineation between these two ideas? If yes, explain why there is no ambiguity. If no, suggest a situation which might be hard to classify.

Yes, I think the delineation is generally clear. I think conflicts of interest stem inherently from selfish gain at the expense of others. If you are overcommitted so you don't meet all expectations, that's very different from intentionally sabotaging a project, manufacturing data, etc. in order to get the results you desire to keep funding from an external funder, for example.

\item Give an example of a conflict of interest which might arise in an academic mentor/mentee relationship?

PhD students work with professors to contribute to research in the lab, and so professors get 
citations and authorships out of those student contributions. Yet once the student graduates, a
professor may lose a collaborator. Therefore, it is somewhat in the interest of the professor to
say that a particular grad student isn't ready to defend their dissertation in order to get more 
publications out of them.

\item Give an example of a conflict of interest which might arise for an author of a published paper.

For statistics in particular, I think there is a certain interest in your method looking good
by beating some sort of benchmark. There are very few papers with negative results, as in "we tried this thing, but it didn't work." Therefore, when designing experiments and simulations, you
may want to choose data that makes your method more appealing than might be practical.

\item You are asked to review a paper for a leading journal. You have high professional respect for the first author, and the paper looks interesting to you. You also count this author among your personal friends. Can you responsibly agree to review the paper? (Imagine you are giving advice to another friend who is in this situation.)

I think you can, but it requires more personal dedication to neutrality. I think it's extremely difficult to critique the work of well respected peers, but at some point, even the most famous statisticians need to be peer-reviewed before publication. The personal friendship part is probably the more difficult, so if there is an option to pass on it, then it may be the more responsible thing, but reviewing a paper for a friend can be fine, especially if the reviewer is anonymous to the reviewee.

\item Most PhD students have to balance time allocated to teaching (GSI) duties with their thesis research. Is this a conflict of interest, or a conflict of commitment, or both? What is your advice on how to manage this balance?

Conflict of commitment, as my research's success isn't caused by the success of my students. To manage this balance, I think it's key to determine the minimum responsibilities required for both and at least dedicate enough time to meet that. After that, you need to spend your time however is most important to you, but... you're not here to be a GSI, you're here to write a dissertation and get the degree.

\item The two main ways to manage conflicts of interest are transparency and avoidance. Give an example of a conflict of interest best managed by avoidance and another best managed by transparency. Explain your answer.

I think avoidance is an easier one - funding from clearly biased agencies or nonprofits should
be avoided in the interest of scientific research, especially when those organizations have clear misalignments with the public interest. However, I recognize the hypocrisy; how does one decide which conflicts are clearly misaligned with public interest and which are not? There may be several organizations funding research whose goals are at odds with one another.

For example, there are organizations who are dedicated to studying climate change for prevention and mitigation, and others that exist to find evidence that climate change isn't happening at all. Now, should scientists be "neutral" and simply be transparent about funding in this case? I would say no. We should have transparency around accepting funding from the NSF to study climate change, but avoidance for taking it from ExxonMobil's think tank, because the former is dedicated to the public interest, and the outcome of the research has no effect on its publication or funding status. In the latter case, I believe there will be very clear pressures to get certain results, and thus public trust in science could be eroded.

\end{enumerate}


\end{document}
