\documentclass[12pt]{article}
\usepackage{fullpage,hyperref}\setlength{\parskip}{3mm}\setlength{\parindent}{0mm}
\begin{document}

\begin{center}\bf
Homework 3. Due by 11:59pm on Sunday 9/21.

Academic misconduct

\end{center}
Read pages 15--23 of {\em On Being a Scientist} and Section 8.1 of the UM Rackham statement on academic misconduct at \url{https://rackham.umich.edu/academic-policies/section8/}. We are going to focus on plagiarism, a subtle topic that has gray areas. Write brief answers to the following questions, by editing the tex file available at \url{https://github.com/ionides/810f25}, and submit the resulting pdf file via Canvas. 

\begin{enumerate}

\item Over several years, I have found it is not unusual when reading STATS 810 homeworks (especially early in the semester) to find responses that include sentences matching the assigned reading word for word, without explicit attribution. Is this plagiarism?

Yes, it is. If you are going to quote a paper of any kind, you should absolutely put the sentences in quotations AND properly cite the source.

\item How do you think a GSI should respond when grading homework which they suspect contains an unattributed cut-and-paste contribution from ChatGPT or any other source?

This is more difficult than it sounds, as there are generally so many submissions it's tough to grade them with any confidence. However, this 
is an "above my pay grade" kind of situation, so I would report it 
to the instructor of record.

\item If you look, you will find common academic practices that are uncomfortably close to plagiarism, if this is strictly interpreted. For example,

(i) Homework problems may be copied or adapted from a textbook, without attribution.

(ii) Figures taken from papers and other internet sources may be presented in talks, class lectures, or GSI lab presentations, without attribution.

Should a responsible researcher attempt to avoid these RCRS gray areas? How?

Ideally, yes, but I think I'm not fully sure what is expected. For example,
if I had to buy a textbook to get homework problems, it seems sort of poor taste to use the homework problems without recommending it to the students,
for example. At the same time, requiring students to buy a textbook is very
expensive. If I were a textbook author, I think I'd be glad people were getting use out of the book, but I'd be pretty annoyed if I ended up
putting in all that work and not making any money off it at all.

The figures issue is odd - I can't imagine doing that. That seems obviously
like plagiarism, and this day and age of having Zotero available, and
other citation managers, I can't imagine why you wouldn't simply err on the
side of citation.

\item Are there any forms of inappropriate scientific conduct that you think have the combination of severity and prevalence to threaten the proper functioning of modern science? Are you more concerned about the total effect of serious (and presumably rare) misconduct, or milder (and potentially more common) misconduct?

I think milder and widespread misconduct are worse. "Salami slicing" is 
something I think has a great opportunity to diminish public confidence
in science, for example. I think there is a public perception that
academics sit in the "ivory tower" and don't provide research that 
reflects the public interest, and so having tons of very small papers 
does little to battle this notion.

\item Self-plagiarism is a subtle topic. When is it acceptable to copy/paste material you have already written into a draft you are currently working on? When is it inappropriate?

If you haven't published it, I believe it's fine. It's less appropriate if you're passing off large chunks of a paper you're already published in a slightly new context. I think there is an element of "what is the goal?" If 
you are self-citing to strengthen a point, then all for it - if it's to
pad a paper, then that's more problematic.

\item Suppose you use AI to help conduct your research, and for drafting or editing your research report. Can this amount to plagiarism? Usually, plagiarism is avoided by clearly attributing the source for each assertion in your writing---do you have advice for how to do this in practice when doing AI-assisted research?

Unsatisfactorily, this sort of comes down to a collective decision that 
may not be made yet for a while. It is a matter of extent, I think - if
code completion is used, for example, I believe this generally doesn't
count as plagiarism in research because code is often somewhat copy
and pasted, and then modified according to need. 

In the writing process, I think the line is clearer; significant amounts
of writing contributed by ChatGPT without attribution passes off someone/something else's work as your own. It should be used sparingly.

At the same time, simply talking with the AI aobut your paper, even asking it for editing advice, seems kind of reasonable. It's the same thing we
do with peers; why not let a machine try that as well? I think similar ethics that govern the contribution of other researchers can be used
as a guideline for others, and in a situation where the contributions
would have led to an authorship, it probably means you're leaning too much on the AI.

\end{enumerate}
\end{document}
