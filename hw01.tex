\documentclass[12pt]{article}
\usepackage{fullpage,hyperref}\setlength{\parskip}{3mm}\setlength{\parindent}{0mm}
\begin{document}

\begin{center}{\bf
Homework 1. Due by 11:59pm on Sunday 9/7.

Building and maintaining healthy mentor/mentee relationships
}\end{center}

Read pages 1--7 of {\em On Being a Scientist}, which is linked on the syllabus, \url{https://ionides.github.io/810f25/syllabus.html}. Write brief answers to the following questions, by editing the tex file available at \url{https://github.com/ionides/810f25}, and submit the resulting pdf file via Canvas. We are particularly concerned with the mentorship relationship between a Statistics PhD student and their thesis adviser at The University of Michigan.

\begin{enumerate}
\item \textit{What roles do mentorship relationships play in professional development of PhD students? }

Mentors start the trajectory of many PhD students' careers. They can provide initial direction, guidance
once topics are decided, and general advice on balancing academic and personal life. Mentors, both intentionally and
not, set the tone for many students - what problems are important, how solutions are found, how writing is 
approached, working styles, \textit{et cetera.}

\item \textit{What do the mentee and mentor gain from the relationship?}

At a practical level, mentors gain research assistance and additional authorship papers. Despite not being first author
on all of them, their names are associated with many more papers. Mentors also gain personal satisfaction
from helping their students succeed, just as someone helped them in the past.

For the mentee, mentors help to focus research and provide essential assistance in technical aspects and 
in the writing process. It's helpful to have a person who is "in your corner" to talk about your best steps
forward, whether that's in journals to submit to, programs to apply to, etc. 

\item \textit{Describe a situation in which the interests of the mentor and mentee are aligned. In RCRS contexts, ``interest'' has a technical definition of benefit or advantage (meaning 11 of \url{https://www.dictionary.com/browse/interest}). Please be careful about this. If you use ``interest'' in the common non-technical meaning (something that excites the curiosity of a person) that is incorrect in an RCRS context such as this question.}

Research publications are prime example - students getting papers published helps with the dissertation writing
and eventual applications for academic or industry positions. Mentors get additional citations as well, but also
more prestige as heads of an active lab or research group.

\item \textit{Describe a situation in which the interests of the mentor and mentee are conflicting. }

Mentors may want additional work from a mentee even though mentees may believe they have fulfilled
their olbigations to the mentor. In this case, when mentors may feel under pressure to publish
for prestige, bonuses, or tenure, it can be tempting to ask more of your mentees. After all, if
more is being asked of you, why shouldn't you be able to ask more of your subordinates?

\item \textit{How are mentorship relationships initiated? E.g., how do you find a thesis adviser?}

Meeting folks in the department. Sometimes mentors will take the initiative to set up these
relationships as well. They can happen organically or more formally, but in either case, 
the mentorship relationships are about shared research interests and personal connections.

\item \textit{Collaboration: What are the advantages and disadvantages for an aspiring statistician of building a mentorship relationship with a researcher who is not a statistician? This could be in addition to, or instead of, having a mentor who is a statistician.}

An advantage of working with a statistician is the direct applicability to your work. If is the most straightforward
form of mentorship, but the directness is also a weakness; statisticians necessarily operatte in an "applied" space,
and so by that token, having more exposure to an outside area of research allows greater "depth" and insight
for future research.

\item \textit{Describe a way in which a mentorship relationship can turn unhealthy. What warning signs should one look for? What actions can one take?}

The primary way that mentor/mentee relationships turn unhealthy is through an abuse of power. Fundamentally, mentors
have a certain amount of power over the future of individuals. This also comes with the responsibility to not abuse
that power, though it can be tempting. Let's say there is a deadline that is coming up as a mentor - do you offload
some of your teaching responsibilities onto a GSI to give you more time? Perhaps you even have your students review
papers that you've been asked to referee; if they say no, there is always the implicit threat to stifle progress toward
the PhD. 

Fundamentally, these relationships rely on a level of trust and good intentions that can be taken advantage of by 
folks who may be themselves suffering from competing stressors and priorities.

\item \textit{A graduate student instructor (GSI) can receive mentorship from the faculty instructor. What are the benefits of building a constructive professional relationship while teaching?}

Teaching is one of the major responsibilities of faculty, and PhDs are generally expected to teach at least some
of the time. Faculty instructors can help guide GSIs in how to be effective teachers and communicators, especially
since most grad students are presumably new to teaching.
  
\end{enumerate}

\end{document}
